%!TeX root = thesis-main.tex
\chapter{Reinforcement Learning}\label{chap:marl}

\minitoc% Creating an actual minitoc
\newcommand{\RS}{\mathcal{S}}
\newcommand{\RA}{\mathcal{A}}
\newcommand{\RP}{\mathcal{P}}
\newcommand{\RR}{\mathcal{R}}
\newcommand{\RE}{\mathbb{E}}

\section{Introduction}
The concept of intelligence is as complex as it is intriguing, 
 and it has been a subject of philosophical inquiry, scientific exploration, and cultural curiosity for centuries. 
 Philosophers have debated on what constitutes intelligence, 
 linking it to \emph{reason}, \emph{wisdom}, and even \emph{morality}. 
%
Despite these varied interpretations, 
 defining intelligence remains a challenge, even in the field of psychology. 
%
Several standardized tests and scales attempt to measure intelligence, 
 but none manage to capture the complete essence of what it means to be ``intelligent''. 
 Intelligence is often understood as the ability to \emph{learn}, \emph{reason}, and \emph{adapt}, among other cognitive abilities.

Among the myriad of perspectives on intelligence, learning stands out as a \emph{fundamental} component. 
 From an evolutionary point of view, the ability to learn is essential for survival. 
 An organism that can adapt to its environment and learn from experiences is likely to survive and reproduce. 
 In the human context, learning has been the cornerstone of development, be it mastering a language, solving complex problems, or creating art.

 This notion of learning is crucial in the realm of Artificial Intelligence (AI). 
%
 If intelligence involves learning, 
 then replicating intelligence artificially would necessarily entail enabling machines to learn. 
 This hypothesis leads us to the exciting and rapidly evolving domain of machine learning---
 a subset of AI that allows computers to learn from data, rather than requiring them to be explicitly programmed for specific tasks.

Machine learning is not monolithic; 
 it encompasses various approaches and techniques that aim to make machines learn. 
 Broadly, these approaches can be categorized into supervised learning, unsupervised learning, semi-supervised learning, and reinforcement learning.
%
Particularly, the first three approaches are based on the idea of learning from data, 
 where the data is either labeled or unlabeled:
\begin{itemize}
  \item \emph{supervised learning}: this is the most straightforward approach, where a model is trained on a labeled dataset. 
  The model makes predictions or decisions based on input data and is corrected when its predictions are incorrect. Typical examples include classification and regression problems;
  \item \emph{unsupervised learning}: unlike supervised learning, this approach does not involve labeled data. 
  The machine tries to learng the patterns and the structure from the data without any supervision (e.g., clustering algorithms);
  \item \emph{semi-supervised}: A middle-ground between supervised and unsupervised learning, this approach utilizes both labeled and unlabeled data for training. The model learns to improve its predictions gradually.
\end{itemize}
Reinforcement learning sets itself apart from other methodologies by operating without the need for labeled data or supervision and through a sequential decisions. 
 It employs a \emph{trial-and-error} strategy, mirroring the way humans acquire knowledge. 
 Subsequent sections will delve into the nuances of this distinctive approach, starting from single agent settings and then moving to multi-agent and many-agent systems---the focus of this thesis.
\section{Single-agent}
\Acl{rl}~\cite{sutton2018reinforcement-learning} is a generic framework 
 inspired by how humans learn
 used to structure \emph{control problems}. 
%
The focus is on \emph{sequential} interactions between \emph{agents}
 (i.e. entity able to act) and an \emph{environment} 
 (i.e. anything outside the control of agents). 
%
At each discrete time step $t$, an agent observes the current environment \emph{state} $s_t$ 
 (i.e. the information perceivable by an agent) and selects an \emph{action} $a_t$ through a \emph{policy} $\pi$ (i.e. a probabilistic mapping between state and action). 
%
Partly as a consequence of its action, at the next time step $t+1$ the agent finds itself in state $s_{t+1}$ and receives a \emph{reward} $r_{t+1}$, 
 i.e. a scalar signal quantifying how good the action was against the given environment configuration. 
%
The goal of \ac{rl} is to \emph{learn} a policy \emph{$\pi^*$} that maximizes the long-term return $G$ (i.e. the cumulative reward) through a \emph{trial-and-error} process. 
%
Different problems can be devised in these settings, like video games~\cite{DBLP:journals/spm/ArulkumaranDBB17}, robotics~\cite{DBLP:journals/ijrr/KoberBP13}, routing~\cite{DBLP:journals/comsur/LuongHGNWLK19}, etc.

This general framework is supported by \ac{mdp}, 
 a mathematical model that describes the environment evolution in sequential decision problems. 
%
A \ac{mdp} consists of a tuple $<\RS, \RA, \RP, \RR>$ in which:
\begin{itemize}
  \item $\RS$ denotes the set of states;
  \item $\RA$ is the set of actions;
  \item $\RP(s_{t + 1} | s_t, a_t)$ define the probability to reach some state $s_{t + 1}$ starting from $s_t$ and performing $a_t$ (i.e. transition probability function);
  \item $\RR(s_t, a_t, s_{t+1})$ devise a probabilistic reward function.
\end{itemize}
In \ac{mdp}, $\RR$ is memory-less, namely the next environment state depends only on the current state. 
%%
Typically, in RL problems, agents do not have access to $\RR$ or $\RP$ but they can rely only on the experience $(s_t, a_t, r_t)$ sampled at a time step. 
%%
Therefore, $G$ is defined as the discounted sum of reward a possible future trajectory $\tau$ (i.e. a sequence of time steps):
%%
\begin{equation}
G_{t} = r_t + \gamma r_{t + 1} + \gamma^2 r_{t + 2} + \dots + \gamma^T r_{t + T} = \sum_{k = t}^T \gamma^{k-t} r_k
\end{equation}
%%
Where $0 \leq \gamma \leq 1$ is the \emph{discount factor}, that is how much the future reward impacts the long-term return.
%%
Finally, the \ac{rl} goal can be expressed as the maximization of the \emph{expected} long-term return following a policy $\pi$:
%%
\begin{equation}
J = \mathbb{E_\pi}\Big[ G_t \Big] = \RE_\pi \Big[ \sum_{k = t}^T \gamma^{t-k} r_k \Big] 
\end{equation}

The \ac{rl} algorithms classification depends on how we derive the $\pi^*$ (i.e. the optimal policy) according to $J$.
%
%\emph{Policy-gradient} methods aim at directly learning the policy through gradient descents algorithms. 
%
In particular, \emph{value-based} methods learn one further function ($Q^\pi$ or $V^\pi$) to derive $\pi^*$.
%
$V^\pi$ is the value function that evaluates how good (or bad) a \emph{state} is according to the long-term return following the policy $\pi$ (\emph{expected value}).
% 
It is defined as:
%%
\begin{equation}
V(s)^\pi = \RE_\pi \Big[ G_t | s_t = s \Big]
\end{equation}
%%
$Q^\pi$ is the corresponding value function that evaluates \emph{state-action} pairs:
%%
\begin{equation}
Q(s, a)^\pi = \RE_\pi \Big[ G_t | s_t = s, a_t = a \Big]
\end{equation}
Policies could be defined through value functions. In particular, a greedy policy based on $Q$ function is the one that always chooses the action with the highest value in a certain state:
\begin{equation}
\pi(s) = \arg \max_{a}(Q(s, a))
\end{equation}
%The objective in value-based methods is to find the best value function (whether $Q$ or $V$), defined as:
%\begin{iequation}
%Q^* = \arg \max_{\pi}(Q^\pi(s, a)) \; \text{or} \; V^* = \arg \max_{\pi}(V^\pi(s))
%\end{iequation}
%Consequently, when we have found $Q^*$, an optimal policy is straightforwardly defined:
%\begin{equation}
%\pi^*(s) = \underset{a}{\text{argmax}}(Q^*(s, a))
%\end{equation}

Q-Learning~\cite{DBLP:journals/ml/WatkinsD92} is one of the most famous value-based algorithms. 
 It aims at finding the $Q^*$ (i.e. the $Q$ function associated with $\pi^*$) by incrementally refining a $Q$ table directly sampling from an unknown environment.
%
Particularly, this is done through a temporal difference update performed at each time step:
\begin{equation}
Q(s_t, a_t) = Q(s_t, a_t) + \alpha *  [r_t + \gamma * \arg \max_{a}(Q(s_{t+1}, a)) - Q(s_t, a_t)]
\end{equation}
Where $\alpha$ is the learning rate (i.e. how much new information will influence the learned $Q$ at each update). 
%
The agent typically follows a $\epsilon$-greedy policy (\emph{behavioural} policy, the function chooses a random action with a $\epsilon$ probability) to balance the \emph{exploitation} and \emph{exploration} trade-off. 

%This algorithm is proven to eventually learn $Q^*$ under the assumption of a \emph{stationary} environment (i.e. the internal dynamics does not change over time).
%
Using $Q^*$ we could extract the $\pi^*$ greedy policy (\emph{target} policy).

Nowadays, Q-Learning is applied in various fields, ranging from robotics to wireless sensor networks and smart grids~\cite{DBLP:journals/access/JangKHK19}.
%
However, one of the most challenging settings in which Q-Learning could be applied is when the learning process has to deal with multiple concurrent learners, namely a multi-agent system.
\section{Multi-agent}
\subsection{Formalization}
In the realm of reinforcement learning, multi-agent systems refer to environments where multiple agents interact with each other and the environment simultaneously. The formalization of multi-agent reinforcement learning (MARL) typically extends the standard Markov Decision Process (MDP) framework to account for multiple agents. Each agent has its own state, action, and reward function, and the joint actions of all agents determine the transition dynamics and rewards.

\begin{equation}
MARL = \langle \mathcal{N}, \mathcal{S}, \mathcal{A}_1, \ldots, \mathcal{A}_N, \mathcal{P}, \mathcal{R}_1, \ldots, \mathcal{R}_N \rangle
\end{equation}

Where:
\begin{itemize}
    \item $\mathcal{N}$ is the number of agents.
    \item $\mathcal{S}$ is the joint state space.
    \item $\mathcal{A}_i$ is the action space of agent $i$.
    \item $\mathcal{P}$ is the joint transition probability function.
    \item $\mathcal{R}_i$ is the reward function for agent $i$.
\end{itemize}

\subsection{Taxonomies}
Multi-agent systems can be categorized based on various criteria:

\begin{itemize}
    \item \textbf{Communication}: Agents can either communicate directly with each other or have no communication.
    \item \textbf{Cooperation Level}: Systems can be fully cooperative, fully competitive, or mixed (both cooperative and competitive).
    \item \textbf{Adversarial Nature}: Some systems have agents with opposing goals, making them adversarial.
\end{itemize}

\subsection{State-of-the-art}
Recent advancements in MARL have led to the development of algorithms that can handle complex interactions between agents. Some state-of-the-art algorithms include Proximal Policy Optimization (PPO) for multi-agent settings, MADDPG (Multi-Agent Deep Deterministic Policy Gradient), and QMIX.

\section{Many-agent}
In particular, in this work, we consider \emph{homogeneous \ac{MAARL}}~\cite{yang2021many}, 
 where the set of agents is large ($N \gg 2$) and each agent is \emph{interchangeable} and \emph{indistinguishable}.
%
This research area is relevant in the context of large-scale systems 
 where collective intelligence emerges from local and repeated interaction of simple entities, like in swarm robotics.
%
In such many-agent scenarios, 
 the implementation of fully decentralized learning is often unfeasible due to the large number of learning agents, 
 which makes the system non-stationary and difficult to manage. 
%
Conversely, a centralized controller capable of coordinating 
 the entire system may not be a viable solution due to scalability concerns. 
 To address this challenge, a practical solution is the adoption of \emph{\ac{CTDE}} approach.
%
The idea is to learn a policy at simulation time when there is a collective view of the system, 
 and then at runtime use that policy but only with local observations. 
%This approach creates policies that are influenced by global information but only require local information to function at runtime. 
%
The typical approach in such cases is based on actor-critic systems~\cite{DBLP:conf/nips/LoweWTHAM17,wu2022more,song2022ctds,song2022centralized},
  where the \emph{actor} is the distributed policy (with only local information) and the critic is a neural network that takes the overall system state.
%
%In fact, the first works in this direction were discussed precisely swarm robotics, %\lukas{what does this mean `in the last field'?}
% exploring new models (e.g., swarMDP~\cite{DBLP:conf/atal/SosicKZK17}) and learning algorithms capable of extrapolating a policy representing the entire system.  \lukas{This sentence still needs rework.}
%Modern approaches, however, 
% have started to consider the use of 
% Recently, deep learning approaches have been considered 
% to synthesise robust controllers capable of generalizing to new tasks. 
%In this context, mean-field reinforcement learning~\cite{pmlr-v80-yang18d} is certainly noteworthy. 
Mean-field RL~\cite{pmlr-v80-yang18d} is one of such concrete applications of \ac{CTDE} 
 where the interactions among the population of agents are estimated by considering either the effect of a single agent and the average impact of the entire population or the influence of neighbouring agents.
%
Some known approaches using mean-field reinforcement learning include Q-mean, 
 which is an extension of Q-learning to mean-field settings~\cite{yang2018mean}, 
 and actor-critic mean-field~\cite{frikha2023actor}, which combines actor-critic algorithms with mean-field approximations. 
%
These approaches have shown promising results in various domains, such as multi-agent coordination 
 and decentralized control, and are actively being researched and developed for further applications.
\subsection{Motivating applications}
Many-agent systems, a subset of multi-agent systems, involve a large number of agents interacting in an environment. These systems are particularly relevant in:

\begin{itemize}
    \item \textbf{Traffic Management}: Managing a large number of vehicles in urban settings.
    \item \textbf{Swarm Robotics}: Coordinating a swarm of robots to achieve a common goal.
    \item \textbf{Financial Markets}: Modeling the behavior of numerous traders in a market.
\end{itemize}

\subsection{Formalization}
\subsubsection{SwarMDP}
A SwarMDP is characterized by a \emph{swarming agent} ($\mathbb{A}$) and the dynamics of the environment ($\mathbb{E}$).
Specifically, $\mathbb{A}$ is a tuple ($\mathcal{S}, \mathcal{O}, \mathcal{A}, \mathcal{R}, \pi$) where:
\begin{itemize}
  \item $\mathcal{S, O, A}$ are the set of local states, observations (or features), and actions, respectively;
  \item $\mathcal{R}: \mathcal{S} \rightarrow \mathbb{R}$ is the reward function, which is influenced by the environment;
  \item $\pi: \mathcal{O} \rightarrow \mathcal{A}$ is the policy function, which maps the observations to the actions: it could be deterministic or stochastic.
\end{itemize}
Starting from this definition, the environment $\mathbb{E}$ is defined as a tuple ($\mathcal{P}, \mathbb{A}, \mathcal{T}, \xi$), where:
\begin{itemize}
  \item $\mathcal{P}$ is the total number of agents in the systems (the agent population), which is assumed to be fixed;
  \item $\mathbb{A}$ is the defined agent prototype that rules each agent $v \in P$;
  \item $\mathcal{T}: \mathcal{S}^P \times \mathcal{A}^P \times \mathcal{S}^P \rightarrow \mathbb{R}$ is the transition  global function, which is influenced by the actions of the agents and returns a collective reward -- this is typically not known by the swarming agents;
  \item $\xi: \mathcal{S^P} \rightarrow \mathcal{O^P}$ is the global observation model of the systems.
\end{itemize}
\subsubsection{Networked Markov Decision Process}

\subsection{State-of-the-art}
In many-agent systems, the sheer number of agents makes traditional MARL techniques computationally infeasible. Recent advancements focus on scalable algorithms and techniques like mean-field theory to approximate the interactions between agents. Some notable algorithms include Mean Field Q-learning and Graph Neural Networks for many-agent systems.

\section{Gran-challenges}
