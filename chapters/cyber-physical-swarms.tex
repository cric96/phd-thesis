%!TeX root = thesis-main.tex
\chapter{Cyber-Physical Swarms}\label{chap:cpsw}
\minitoc% Creating an actual minitoc
The aim of this section is to elucidate the concept of \textit{Cyber-Physical Swarms}, 
 distinguish systems that align with our conceptual framework, 
 and identify research domains grappling with analogous challenges. 
% 
\section{Overview}
The term refers to a network of computational nodes working in concert to address collective problems, 
 akin to naturally occurring swarm phenomena.

The architecture of these systems comprises a large array of interconnected nodes, 
 numbering in the hundreds or thousands. 
% 
Importantly, these networks are \textit{open}, 
 implying that the total node count is not predetermined. 
% 
As a result, 
 the emergent collective behaviour should exhibit \textit{scale-independence}, 
 ensuring that the same programmatic logic applies across small and expansive networks alike.

In these systems, each node is tethered to a \textit{physical} entity through sensors and can influence its environment via actuators. 
 This integration of computational and physical elements justifies the term ``cyber-physical''. 
 The nodes may exhibit heterogeneity in hardware or other attributes, 
 yet they maintain a \emph{homogeneous} local behaviour, executing identical local algorithms.

Nodes operate with a focus on collective goals, 
 as opposed to individualistic or selfish objectives, 
 thereby adhering to a collaborative ethos. 
% 
Given that centralization is not feasible, 
 the architecture relies on distributed control mechanisms. 
 While modern computing paradigms such as cloud or edge infrastructures are applicable, 
 it is imperative that the system be capable of swift local responses. 
 This is especially critical given that these are often \emph{critical systems}, 
 where delays in decision-making could have severe repercussions.

In concluding, 
 it is pertinent to clarify that the focus of this study is on the \textit{macro-level} behaviours of the swarm, 
 rather than the \textit{micro-level} interactions. 
 Rather than characterizing the collective through emergent properties, 
 the intention is to define a globally desirable structure.

As for related concepts, 
 \acp{cpsw} can be considered a subset of Complex Adaptive Systems~\cite{holland1992complex}. 
 However, our framework imposes additional conditions not universally found in standard definitions of complex systems, 
 such as many collaborative agents.
% 
Collective Adaptive Systems~\cite{DBLP:journals/corr/abs-1108-5643} closely resemble our definition of \acp{cpsw}, 
 but they diverge in that agents in our framework are geared toward \emph{collective} goals 
 and exhibit homogeneous behaviour.

\section{Vision examples}
\subsection{Wild-fire monitoring in extensive forests}
Canada is renowned for its lush forests, 
 which cover approximately one-third of the nation's landmass, 
 making it one of the countries with the largest forested areas globally. 
These green expanses are not only a source of natural beauty but also play a crucial role in maintaining ecological balance. 
 However, the increasing impact of climate change has led to a surge in wildfires, 
 causing widespread devastation to both local flora and fauna.
 Consider for instance the 2023 wildfires, 
 which burned over 43 million acres, 
 that is about 5\% of the entire forest area of Canada~\cite{enwiki:1178342069}. 

To address this pressing issue, 
 there is a growing consensus on the need for proactive and localized interventions. 
 Specialized monitoring systems are essential for keeping tabs on vulnerable regions, 
 especially given the impracticality of relying solely on human observation due to the vastness of these areas. 
 One innovative solution might be the deployment of a sophisticated network of environmental sensors, strategically placed to monitor temperature, humidity, and other fire-prone conditions. 
 These sensors are complemented by a swarm of drones equipped with advanced imaging and data collection capabilities.

The system operates through ongoing collaboration between ground-based sensors and aerial drones. 
 The drones provide a bird's-eye view of the landscape, 
 allowing for real-time monitoring of a wide array of nodes across large geographical expanses. 
 This integrated approach enables the early detection of potential fire hazards, 
 thereby facilitating the pre-emptive mobilization of emergency services.

However, the implementation of such a comprehensive monitoring system comes with its own set of challenges. 
 Each sensor or drone has a limited operational range and can only provide a \emph{partial} view of the overall system. 
 Therefore, robust data fusion algorithms are necessary to merge information from multiple sources into a cohesive and actionable overview. 
 Environmental conditions are also highly dynamic, requiring the system to \emph{adapt} in real-time to changing variables such as wind speed, temperature fluctuations, and precipitation levels.

In specific regions where the risk is elevated
 -- such as areas with active fires or reduced visibility due to smoke or fog—there may be a need for deploying additional sensors and drones. 
 Given that drones have limited battery life and need to be recharged, 
 the system must be capable of \emph{self-organizing} to ensure uninterrupted monitoring. 
 This is particularly challenging considering that the number of deployed devices could easily exceed thousands of units across the extensive area of interest.

Moreover, the system must be highly responsive to emergency conditions. 
 This necessitates that each device, whether a sensor or a drone, 
 should be equipped with \emph{edge computing} capabilities for performing local analyses. 
 These local analyses can then be used to trigger collective alarms, 
 ensuring immediate action is taken to mitigate the risk of wildfires.
\subsection{Crowd steering}
Concert venues (or public event like a soccer match) 
 are often filled with an energetic and enthusiastic crowd, 
 eager to enjoy live performances. 
 These events can attract tens of thousands of attendees, 
 making crowd management a critical concern for both safety and enjoyment. 
% 
However, traditional methods of crowd control, such as barriers and security personnel, 
 are increasingly proving to be inadequate in the face of evolving challenges like sudden surges or emergency situations. 
 Take, for example, the 2023 Summer Music Festival, 
 where a sudden downpour led to chaotic movements among the crowd, 
 resulting in several minor injuries~\cite{concertsafety:2023report}.

To address this complex issue, 
 there is a growing interest in leveraging technology for more effective crowd steering. 
 One promising approach is to equip each attendee with smart bracelets or utilize their smartphones, 
 both of which have computational capabilities. 
 These devices can communicate with a centralized system as well as with each other, 
 forming a dynamic and adaptive network.

The system functions through real-time data exchange between the individual devices and strategically placed sensors around the venue. 
 These sensors can monitor various parameters such as crowd density, 
 movement speed, and even noise levels. 
 The individual smart devices provide a ``ground-level'' perspective, 
 allowing for a more granular understanding of crowd behaviour.

This collaborative approach enables the system to identify potential issues before they escalate. 
 For instance, if a particular section of the venue becomes too crowded, 
 the system can send alerts or directions to the smart devices in that area, 
 advising attendees to move to less crowded sections. 
 This facilitates the proactive redistribution of the crowd, thereby averting potential safety hazards.

However, implementing this advanced crowd steering system is not without challenges. 
 Each smart device only has a \emph{limited} computational capacity and can provide just a \emph{partial} view of the overall crowd dynamics. 
 Therefore, sophisticated algorithms are needed to aggregate this fragmented data into a comprehensive and actionable overview. 
 Additionally, the system must be able to \emph{adapt} rapidly to changing conditions, such as sudden weather changes or unexpected incidents during the concert.

In specific situations where immediate action is required -- such as medical emergencies or security threats -- there may be a need to send targeted alerts or instructions to specific groups of attendees. Given that smart devices have limited battery life, 
 the system must also \emph{self-organize} to prioritize critical alerts and instructions, 
 ensuring that the crowd management remains effective throughout the event.

Moreover, the system must be highly responsive to real-time conditions. 
 This necessitates that each smart device should be equipped with \emph{edge computing} capabilities, allowing for localized data processing and decision-making. 
 These local analyses can then trigger collective actions, such as coordinated movements or emergency evacuations, ensuring that immediate and effective measures are taken to maintain crowd safety.

Through the integration of technology and real-time data analytics, 
 concert organizers aim to create a safer and more enjoyable experience for attendees, 
 effectively steering large crowds and responding proactively to any challenges that may arise.
\section{Swarm Intelligence}

Swarm Intelligence tries to apply the tactics observed in social animals in Swarm Robotics. 
%
The behaviours are derived in a bottom-up fashion, hence designers have tried to achieve collective behaviours through emergence observing individual animal behaviours.
%
For instance, artificial stigmergy~\cite{DBLP:journals/fgcs/DorigoBT00} derives from those studies.
%
However, nowadays, Swarm Intelligence is more focused on algorithms.
% 
Indeed, the collective behaviours observed in nature are leveraged to perform \textit{optimization} strategies or to directly solve problems by searching in the solution space.
%%
There are several examples of Swarm Intelligence optimisation algorithms, such as Ant Colony Optimisation (ACO)~\cite{DBLP:journals/tsmc/DorigoMC96}, Particle Swarm Optimisation (PSO)~\cite{DBLP:conf/icnn/KennedyE95} and Flock of Starling Optimisation (FSO)~\cite{DBLP:series/sci/FulgineiS11}.

Even if they are important approaches, I am not directly interested in using them. 
%
Indeed, I draw \textit{inspiration} from swarms but only to achieve similar behaviour in the artificial swarm using self-organization. 
%
I do not want to \textit{mimic} nature but I aim to exploit the same mechanisms to build robust collective systems.

\section{Swarm Robotics}

Historically, Swarm Robotics derives from the first Swarm Intelligence approaches. 
%
Then, it has emerged as the \textit{engineering} part of that branch. 
%
Indeed, the general goal of \emph{swarm engineering} is \emph{to define systematic and well-founded procedures for modelling, designing, realising, verifying, validating, operating, and maintaining a swarm robotics system}~\cite{DBLP:journals/swarm/BrambillaFBD13}.

In Swarm Robotics, the focus is mainly on \textit{robots} that are \emph{autonomous}, \emph{situated}, and with \emph{no central control}. 
%
I aim to expand this vision also to other ``swarm-like'' systems, such as a crowd of people, large-scale IoT and smart cities where I do not have robots. 
%
In particular, the novel branch of \textit{automatic design}~\cite{DBLP:journals/firai/FrancescaB16} is very appealing to be used in \acp{cpsw}. 
%
In autonomic design approaches, the controller is derived through \textit{genetic algorithms} or \textit{Multi-agent Reinforcement Learning} following a \textit{global} utility function. 
\subsection{Multi-Agent Systems}
A \ac{cpsw} could be seen as a Multi-Agent System -- in particular a \emph{many}-agent system -- where a group of autonomous entities are programmed to achieve collective behaviours through \emph{repeated} sensing, computation, communication, and actuation.

Due to the high stochasticity of the environment, it is almost impossible to know and program in advance the optimal behaviour for all agents.
%
%\printbibliography