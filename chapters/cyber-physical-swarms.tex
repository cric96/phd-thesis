
\chapter{Cyber-Physical Swarms}\label{chap:cpsw}
\minitoc% Creating an actual minitoc

The goal of this section is to clarify what I mean by \textit{Cyber-Physical Swarms}, what systems are similar to our definition, and which research areas have to deal with similar problems.
This term is used to describe many computational nodes that collaborate to solve collective problems, similarly to what we observe in natural ``swarm'' systems.
%
These systems have a large population of networked nodes (hundreds -- thousands), and they are typically \textit{open} (i.e. we cannot know the total node number apriori). 
%
Consequently, the collective behaviour should be \textit{scale-independent} (i.e. the same program must work in small networks as well as in very large networks).
%
Besides, nodes are associated with a \textit{physical} asset through sensors and could change the environment using actuators. 
%
Namely, they have an \emph{embodiment} in the world -- so the name of cyber-physical. 
%
The nodes could be heterogeneous, but with a \emph{homogenous} local behaviour (i.e., each node should execute the same local logic). 
%
Furthermore, nodes act collaboratively, and they are not selfish -- they always operate for the global collective good.
%
I suppose that the overall system could not use a central entity -- so I need to deal with a distributed control. 
%
Even if modern architectures could be used (e.g cloud/edge infrastructures), in general, the system needs to react to local problems quickly. 
%
Hence, moving this decision far from the physical assets could be dangerous (since they are a kind of \emph{critical systems}).
%
As a concluding note, I want to consider their behaviour at the \textit{macro-level}, not at the \text{micro-level}. 
%
Hence, I intend to describe the collective not by emergence, but by defining a global wanted structure.

\acp{cpsw} are kind of Complex Adaptive systems~\cite{holland1992complex}, but our vision enforces properties that are not necessarily present in standard complex systems' definition, like the large agents' number and the collaborative agents' nature.
%%
Collective Adaptive Systems~\cite{DBLP:journals/corr/abs-1108-5643} are near to our \acp{cpsw} description but in our definition agents pursue \emph{collective} (not individual) goals, and they have a homogenous behaviour.

\section{Swarm Intelligence}

Swarm Intelligence tries to apply the tactics observed in social animals in Swarm Robotics. 
%
The behaviours are derived in a bottom-up fashion, hence designers have tried to achieve collective behaviours through emergence observing individual animal behaviours.
%
For instance, artificial stigmergy~\cite{DBLP:journals/fgcs/DorigoBT00} derives from those studies.
%
However, nowadays, Swarm Intelligence is more focused on algorithms.
% 
Indeed, the collective behaviours observed in nature are leveraged to perform \textit{optimization} strategies or to directly solve problems by searching in the solution space.
%%
There are several examples of Swarm Intelligence optimisation algorithms, such as Ant Colony Optimisation (ACO)~\cite{DBLP:journals/tsmc/DorigoMC96}, Particle Swarm Optimisation (PSO)~\cite{DBLP:conf/icnn/KennedyE95} and Flock of Starling Optimisation (FSO)~\cite{DBLP:series/sci/FulgineiS11}.

Even if they are important approaches, I am not directly interested in using them. 
%
Indeed, I draw \textit{inspiration} from swarms but only to achieve similar behaviour in the artificial swarm using self-organization. 
%
I do not want to \textit{mimic} nature but I aim to exploit the same mechanisms to build robust collective systems.

\section{Swarm Robotics}

Historically, Swarm Robotics derives from the first Swarm Intelligence approaches. 
%
Then, it has emerged as the \textit{engineering} part of that branch. 
%
Indeed, the general goal of \emph{swarm engineering} is \emph{to define systematic and well-founded procedures for modelling, designing, realising, verifying, validating, operating, and maintaining a swarm robotics system}~\cite{DBLP:journals/swarm/BrambillaFBD13}.

In Swarm Robotics, the focus is mainly on \textit{robots} that are \emph{autonomous}, \emph{situated}, and with \emph{no central control}. 
%
I aim to expand this vision also to other ``swarm-like'' systems, such as a crowd of people, large-scale IoT and smart cities where I do not have robots. 
%
In particular, the novel branch of \textit{automatic design}~\cite{DBLP:journals/firai/FrancescaB16} is very appealing to be used in \acp{cpsw}. 
%
In autonomic design approaches, the controller is derived through \textit{genetic algorithms} or \textit{Multi-agent Reinforcement Learning} following a \textit{global} utility function. 
\subsection{Multi-Agent Systems}
A \ac{cpsw} could be seen as a Multi-Agent System -- in particular a \emph{many}-agent system -- where a group of autonomous entities are programmed to achieve collective behaviours through \emph{repeated} sensing, computation, communication, and actuation.

Due to the high stochasticity of the environment, it is almost impossible to know and program in advance the optimal behaviour for all agents.
%
%\printbibliography