%!TeX root = thesis-main.tex
\chapter{Conclusion}\label{chap:wrap-up}%\mtcaddchapter
This dissertation situates itself within the realm of complex systems engineering and \acf{MAARL} applications, 
 adopting unconventional methods to address the unique challenges posed by \acf{cpsw} (introduced in ~\Cref{chap:cpsw}). 
 It is grounded in a language-based approach focused on aggregate computing (discussed in \Cref{chap:macro-programming}), 
 further integrated with machine learning techniques (overviewed in \Cref{chap:marl})---particularly the one based on the notion of \emph{many-agent}. 

Focusing on the research questions posed in \Cref{chap:introduction}, 
 we have sought to address the following.
Regarding \textbf{RQ1}, 
   we have chosen to use the aggregate computing model 
   as a reference because it allows for
   handling the various challenges of CPSW, 
   including scale, collective behaviors, and the possibility of creating self-healing behaviors. 
   We have also shown other plausible models used in different contexts, such as the multi-agent RL, particularly SwarMDP. 
   We have seen how these can be combined (see \Cref{chap:rl:field-informed}).
 
About \textbf{RQ2}, 
  we explored the various aspects of this combination (i.e., the hybrid approach) in \Cref{part:learning}. 
  We demonstrated how this seems to be a winning combination at all
  levels of engineering, 
  both at the application level and the platform level.
 
 For \textbf{RQ3}, 
  we sought to outline the main characteristics that differentiate CPSW
  systems from known ones --- see discussion in \Cref{chap:cpsw}. 
  In particular homogeneity and self-organisation were the characteristics that then
  guided the choice of the reference model and the solutions discussed.
 
Finally, concerning \textbf{RQ4}, 
 we discussed throughout the thesis how this
 influences the design process, 
 discussing new patterns and how this affects the
 distributed execution platform of such applications. 
 
Following this, 
 an in-depth analysis of the principal contributions 
 and potential future work that continues in the direction proposed by this thesis is presented.
\section{Discussion}
\subsection*{Design Patterns for Cyber-Physical Swarms}
\sloppy
\ac{ac} proves to be a composable, 
 modular, and predictable approach for self-organizing systems, 
 making it versatile across various domain-specific applications.
% 
After identifying the class of \ac{cpsw}, 
 we noted existing gaps in aggregate computing's ability to manage the complexity inherent in these systems. 
 To address this, we first identified algorithms capable of supporting high-level applications, 
 such as the swarm clustering algorithm introduced in \Cref{chap:eng:clustering}. 
% 
We further sought to identify design patterns that can aid designers in crafting collective processes, 
 as detailed in \Cref{chap:eng:decentralization}. 
 Lastly, we encapsulated an essential API for coordinated movement and collective decision-making in \Cref{chap:eng:macroswarm}. 
%
This corpus of contributions serves as a milestone in designing applications for \acp{cpsw}, 
 providing designers with both a suite of essential tools and a design space that narrows the gap between the problem space and the solution space.
\subsection*{Deployments in Complex Infrastructure}
Aggregate computing offers a flexible model with minimal constraints on deployment. 
 In this dissertation, 
 the model aligns perfectly with the \acp{cpsw} systems under consideration. 
 However, this top-down approach eventually needs to be grounded, 
 ensuring that the collective application remains independent of the chosen deployment scheme, 
 thereby decoupling logic from architectural considerations.
%
To that end, 
 we introduced a modern deployment approach for collective applications based on the ``pulverization'' model and multi-tier programming (\Cref{chap:eng:multitier}). 
 This allowed us to capture essential components required for aggregate computation, 
 facilitating their opportunistic distribution based on the chosen infrastructure, 
 managed through the multi-tier paradigm, 
 particularly leveraging Scala Loci.

\subsection*{Novel Programming Models for CPSws}
Field calculus emerged as an effective paradigm for declarative and expressive conceptualization and implementation of complex, 
 self-organizing behaviours at the collective level. 
% 
This was applied in a range of applications, 
 from swarm robotics and crowd engineering to data centres.
% 
However, the approach highlighted certain limitations,   particularly concerning \emph{efficiency}--
 crucial in aggregate systems featuring low-power computational devices. 
 In response, we introduced FRASP (\Cref{chap:eng:frp}), 
 a new model founded on reactive programming combined with spatial computing, offering greater flexibility in execution by allowing partial re-computation when needed.

\subsection*{Learning for Aggregate Computing}
Our language-based approach opened a new frontier in the design of collective applications. 
 Conventionally, a stark distinction exists between manual design (e.g., defining behaviours through programming languages) and automatic design (e.g., employing machine learning techniques for collective behaviours). 
% 
In this thesis, we attempted to bridge these two worlds,
 striking a balance between declarative simplicity and the complexity and adaptability of described collective behaviours. 
%
 To do so, we outlined a roadmap (\Cref{chap:learning:roadmap}) that helped us frame the problem and the layers of interaction required for this hybrid vision.
% 
 Our research ventured into program synthesis in collective systems, 
 where parts of the program were left undefined to be later filled by \ac{MAARL} algorithms (\Cref{chap:learning:sketching}). 
% 
Another line of work focused on learning intelligent scheduling policies to expedite the attainment of equilibrium compared to a pre-set program (\Cref{chap:rl:schedulers}).
%
Lastly, we introduce a novel tool called ScaRLib (\Cref{chap:rl:scarlib}), 
 which allows for the seamless integration of aggregate computing and deep reinforcement learning, 
 supporting the hybrid design discussed in this thesis.
\subsection*{Aggregate Computing for Learning}

In line with the hybrid approach, aggregate computing itself can serve as a medium for enhancing learning. As explored in \Cref{chap:rl:field-informed}, 
 we combined graph neural networks with aggregate computing and deep reinforcement learning. 
 This unification proved to be highly effective, 
 allowing for localized learning processes that are informed by collective knowledge gathered through aggregate computing, akin to programmed shimmery.

\section{Future works}

\subsection*{Unified programming model}
In the quest for a seamless interface between aggregate computing, novel reactive models, and machine learning, 
 the development of a unified programming model stands as a crucial next step.
% 
This framework could enable practitioners to design complex behaviours across the swarm with reduced cognitive load, 
 focusing on higher-level objectives instead of intricate details. 

\subsection*{Comprehensive Swarm API}
While the presented patterns in this thesis serve as a good foundation, it is by no means exhaustive. 
 Upcoming work aims to expand the API to accommodate more swarm behaviours and complex orchestrations, 
 further streamlining the application development process for cyber-physical swarm systems. 
 This will entail not only algorithmic contributions but also real-world testing and validation against robustness and scalability criteria.

\subsection*{Benchmarking}
As more algorithms and approaches get introduced in this space, 
 benchmarking them against standard metrics becomes essential for meaningful comparison and adoption.
% 
Future efforts will aim to establish such benchmarks, 
 enabling the empirical evaluation of different strategies in terms of performance, efficiency, and adaptability.

\subsection*{Opportunistic Deployment}
Building on the deployment methods discussed,
 future work will also investigate opportunistic deployment strategies that take advantage of the system's inherent flexibility. 
 For example, 
  identifying the conditions under which it might be beneficial to move computation closer to the data source or a more powerful computational node, 
  thereby optimizing for both efficiency and latency.

\subsection*{Aggregate Computing for Learning: Distributed Learning}
Following the promising results in leveraging aggregate computing for learning, 
 the next steps will focus on enhancing distributed learning paradigms. 
 This can open avenues for more effective, decentralized learning schemes (e.g., federated learning), 
 reducing the need for a centralized learning authority and thereby increasing the robustness and resilience of the system.

\subsection*{Online Learning}
The current research has largely focused on offline learning mechanisms. 
 However, real-world applications often require the ability to adapt and learn in \emph{real time}. 
 Future work aims to extend the proposed models and algorithms for online learning capabilities, 
 allowing swarms to dynamically adapt to changing environments.

\subsection*{Learn to Deploy}
Taking inspiration from the emerging ``DevOps'' culture in software engineering, 
 we plan to explore how learning algorithms can aid in the deployment and management of swarm applications. 
 This line of research would aim to automate many aspects of deployment, 
 from selecting optimal locations for computational nodes to dynamically allocating resources based on real-time system performance.

\subsection*{Unified Design Process}
Last but not least, 
 integrating all these disparate elements--
 programming models, deployment strategies, and learning algorithms---into a unified design process represents a grand challenge for future work. 
 The goal would be to offer practitioners a one-stop solution for designing, deploying, and managing cyber-physical swarm systems effectively and efficiently.