\chapter{\introductionname}\label{chap:introduction}
The recent evolution of IT technologies fosters a vision in which computation is \textit{everywhere}. 
 This evolution is marked by computational capabilities seamlessly woven into the fabric of daily life, 
 transcending traditional boundaries. 

Modern paradigms such as ubiquitous~\cite{DBLP:journals/sigmobile/Weiser99}, pervasive
 ~\cite{DBLP:journals/computer/SahaM03}, collective~\cite{DBLP:journals/computer/Abowd16}, 
 and automatic computing~\cite{DBLP:journals/computer/KephartC03} champion this transformative vision. 
%
These paradigms pivot around \ac{cas} where vast arrays of simple devices interact in a decentralized fashion, 
 harnessing their collective might to accomplish intricate tasks. 
%
Drawing parallels from nature, 
 these systems mirror the \textit{complexity} inherent in natural systems, particularly those involving collective behaviours.

In the realm of nature, certain social animals, such as insect swarms, 
 demonstrate resilient, effective, and efficient collective behaviours through self-organization mechanisms. 
 Taking inspiration from this, the objective of this thesis is to imbue \ac{cas} with characteristics akin to those found in natural swarms. 
%
This has led to the conceptualization of \ac{cpsw}: 
 computational entities intrinsically bound to the physical world, 
 achieving collective objectives through innate self-organizing tendencies. 
%
Swarm robotics, human crowds, and, more broadly, IoT device swarms exemplify instances of \acp{cpsw}.

However, traditional device-centric design methodologies falter when confronted with the unique challenges posed by \acp{cpsw}. 
 These challenges include issues arising from the nuanced local-global interplay, 
 the intricacies of distributed control, 
 burgeoning IT architectures, 
 and overarching scalability concerns. 
%
Central to this research endeavour is the pursuit of a systematic approach encompassing robust models, 
 innovative techniques, 
 and pioneering algorithms that facilitate the synthesis and deployment of predictable self-organizing behaviours within \acp{cpsw}.

Historically, the design and study of such systems were inspired by observations of natural phenomena. 
 This bottom-up approach, although insightful, often culminated in specialized solutions with limited scalability. 
 Contrasting this, contemporary methodologies, as advocated in this research, promote top-down strategies. 
 Here, system outcomes are envisioned at the collective echelon, ensuring broader applicability and scalability. 
 Notable frameworks in this space include Buzz~\cite{DBLP:journals/software/PinciroliB16} and TOTA~\cite{DBLP:conf/icdcsw/MameiZL03}. 
%
This thesis, however, will primarily focus on \acfi{ac}~\cite{DBLP:journals/computer/BealPV15}. 
 The choice is motivated by \ac{ac}'s potential for defining functionally composable self-organizing behaviours, 
 paving the way for versatile aggregate programs that are domain-agnostic and scale gracefully with complexity.

While \ac{ac} has already found application in diverse domains, 
 such as crowd dynamics, urban infrastructures in smart cities, and extensive IoT networks, 
 there exists a significant gap in its foundational framework. 
%
This deficit spans areas like robust middleware, systematic program articulation, intricate abstraction layers, and core principles.
%
Addressing this lacuna, this thesis sets out to rigorously investigate and critically appraise \ac{ac} methodologies within the context of \acp{cpsw}. 
%
This journey promises to traverse myriad avenues encompassing distributed intelligence, adaptive middleware solutions, and specialized building blocks, with aspirations to contribute significantly to both foundational and architectural realms.

\section{Publication List}
