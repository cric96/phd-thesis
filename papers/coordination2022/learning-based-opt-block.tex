%!TeX root = paper22-coord-ac-rl.tex

\begin{lstlisting}[
  mathescape,
  %float=tp,
  floatplacement=tbp,
  frame=single,
  label={coordination2022:lst:general-schema},
  caption={
    ScaFi-like pseudocode description (implemented in the simulation) for value-based \ac{rl} algorithm applied \ac{ac}.
    %
    \texttt{state}, \texttt{update}, \texttt{reward} are block specific. 
    %The learn branch use \texttt{simulation}, a global object in which agents access to a shared Q.
  },
  captionpos=b
]
def optBlock($o_{t-1}$) { // learning as a field that evolves in time
  rep(($s_0, a_0$, $o_0$)) { // $s_0$, $a_0$ context dependent 
    case ($s_{t-1}, a_{t-1}$, _) => {
      val Q = sense("Q") // global during training, local during execution
      val $o_{t}$ = update($o_{t-1}$, $a_{t-1}$) // local action
      // state from the neighbourhood field program output
      val $s_{t}$ = state(nbr($o_{t}$))
      val $a_{t}$ = branch(learn) { // actions depends on learn condition
        val $r_{t-1}$ = reward($o_{t}$, simulation) // simulation is a global object
        simulation.updateQ(Q, $s_{t-1}$, $a_{t-1}$, $r_{t-1}$, $s_{t}$) // Q update
        $\sim$ $\pi_{behavioural}^Q$($s_{t}$) // sample from a probabilistic distribution
      } {
        $\pi_{target}^Q$($s_t$) // greedy policy, no sampling is needed
      }
    }
    ($s_{t}$, $a_{t}$, $o_{t}$) 
  }._3 // select the output from the tuple
}
  \end{lstlisting}
  %\caption{\scafi{}-like pseudocode description for value-based \ac{rl} algorithm applied \ac{ac}. \texttt{state}, \texttt{update}, \texttt{reward} are block specific. The learn branch use \texttt{simulation}, a global object in which agents access to a shared Q. 
  %$\sim$ stands for sampling a value from a probabilistic distribution.}
  %\label{lst:general-schema}
%\end{figure}