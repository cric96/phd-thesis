\documentclass{article} 
\usepackage[usenames,dvipsnames,svgnames,table]{xcolor}
\usepackage{hyperref}
\usepackage[usenames]{xcolor}
\usepackage{graphicx}
\usepackage{amsmath}
\usepackage{listings}
\usepackage{nameref}

\RequirePackage[sort&compress,sectionbib,numbers]{natbib}


\def\BibTeX{{\rm B\kern-.05em{\sc i\kern-.025em b}\kern-.08em
    T\kern-.1667em\lower.7ex\hbox{E}\kern-.125emX}}

\usepackage{xr}
\externaldocument[main:]{../paper20-mdpi-jsan-si-autonomy}
\usepackage{xr-hyper}


\newcounter{reviewer}
\newcounter{comment}[reviewer]

\setcounter{reviewer}{-1} % start with zero
\newcommand{\reviewer}{
	\subsection*{\refstepcounter{reviewer}Reviewer \arabic{reviewer}}
}
\newcommand{\reviewerl}[1]{
	\subsection*{\refstepcounter{reviewer}\label{#1}Reviewer \arabic{reviewer}}
}
\newcommand{\reviewerlt}[2]{
	\subsection*{\refstepcounter{reviewer}\label{#1}#2}
}
\newcommand{\comment}[1]{
	\subsubsection*{\refstepcounter{comment}Reviewer comment \arabic{comment}} %\arabic{reviewer}.
	\colorbox{gray!10}{\parbox[t]{\linewidth}{\setlength{\parskip}{0.5\baselineskip}%
 #1 }}
}
\newcommand{\rcref}[2]{\ref{#1}.\ref{#2}}
\newcommand{\commref}[1]{\ref{#1}}
\newcommand{\commentl}[2]{
	\subsubsection*{\refstepcounter{comment}\label{#1}Comment \arabic{comment}} %\arabic{reviewer}.
	\colorbox{gray!10}{\parbox[t]{\linewidth}{ #2 }}
}
\newcommand{\edcomment}[1]{
	\subsubsection*{Comment}
	\colorbox{gray!10}{\parbox[t]{\linewidth}{ #1 }}
}
\newcommand{\reply}[1]{	\\[2pt]
	\textbf{Author response:} 	
	#1
}
\newcommand{\action}[1]{	\\[2pt]
	\textbf{Action to address the comment:} 
	#1
}
\newcommand{\corrstart}{\color{red}}
\newcommand{\corrend}{\color{black}}
\newcommand{\correction}[1]{\corrstart #1\corrend{}}
\newcommand{\checkstart}{\color{violet}}
\newcommand{\tocheck}[1]{\checkstart{}#1\corrend{}}

\makeatletter
\renewcommand\p@comment{\thereviewer.}
\makeatother

\newcommand{\say}[3]{
\noindent\fcolorbox{black}{cyan!20!white}{
\begin{minipage}{\textwidth}
\textbf{{#1}} \textbf{({@}#2)}: #3
\end{minipage}
}
}


\usepackage{cleveref}


\begin{document}

\title{{\Large Revision Letter for} ``A Programming Approach to Collective Autonomy''}
\author{
Roberto Casadei
\and
Gianluca Aguzzi
\and
Mirko Viroli
}

\maketitle

Dear Reviewer, \newline

Thank you very much for your time and valuable feedback.
%
We have taken into account all your questions and suggestions and implemented corresponding revisions to our manuscript.
%
Our replies and corrective actions for your comments are detailed below.
%
Based on your comments and the comments raised by other Reviewers, we have implemented the following main revisions:
%
\begin{itemize}
\item we have added a comparison with respect to related work (\Cref{main:s:summary-comparison-rw});
\item we have clarified the contribution
(\Cref{main:s:intro}, \Cref{main:s:conc});
\item we have run more experiments and clarified aspects of the evaluation (\Cref{main:s:eval});
\item we have fixed several minor issues to further improve the overall quality of the manuscript.
\end{itemize}
%
For the Reviewer's convenience, we coloured in {\color{blue}blue} the parts of the paper that have been added or modified.
%
We hope that you will be satisfied with the revised version of our paper and consider it now suitable for publication.
\\ ~ \\

\noindent Sincerely yours,

Gianluca Aguzzi, Roberto Casadei, Mirko Viroli


\raggedbottom

%\section*{Comment and replies}

%\reviewerlt{r0}{Editor's Comments}
%\edcomment{
%}
%\reply{
%}

% (x) English language and style are fine/minor spell check required


%	Yes	Can be improved	Must be improved	Not applicable
%Does the introduction provide sufficient background and include all relevant references?
%( )	(x)	( )	( )
%Is the research design appropriate?
%( )	(x)	( )	( )
%Are the methods adequately described?
%( )	(x)	( )	( )
%Are the results clearly presented?
%( )	(x)	( )	( )
%Are the conclusions supported by the results?
%( )	(x)	( )	( )



\reviewerlt{r4}{Comments and Replies}

\comment{
In this paper, the authors suggest the followings

\begin{itemize}
\item we provide a review of literature about autonomy and especially collective autonomy in MASs;
\item we analyse the aggregate computing framework by the perspective of autonomy, by covering its positioning w.r.t. individual and collective autonomy, and showing how it can supports adjustable autonomy;
\item we exemplify the discussion through a simulated case study, investigating (i) the relationship between individual goals/autonomy and collective goals/autonomy; and (ii) the relationship between structures and collective autonomy; and ù
\item we provide gaps in literature on programming reliable collective autonomy and delineate a research roadmap.
\end{itemize}
This paper is generally well-written, but needs some supplements.
}
\reply{ Thanks for the appreciation and for the suggestions, that we have carefully considered to carry out the current revision. }

\comment{
=\textgreater There are many types of 'autonomy'. It is better for the author to summarize and show them in a table.
}
\action{ 
A new table, \Cref{main:table:autonomy-notions}, has been added to summarise the various notions of autonomy that have been mentioned in the manuscript.
}


\comment{
=\textgreater In the simulation results, there are some cases where p \textless 1 is better. What does this mean?. 
In addition, there seems to be a need for more simulations to support the proposition.
}
\reply{ 
Thanks for pointing out this aspect of the results, which needs a clarification.
%
Further experiments have revealed that the differences 
 highlighted in curves are not significant (i.e., they are due to a ``sampling error'').
 Essentially, the metric ``healed count'' should not be considered absolutely but rather relatively to the number of problems to be solved that emerge.
 In other words, the ``healed count'' can only be considered as an approximation of the system performance, which is better described by other means (see below).
}
\action{ 
We added a measure of error that summarily represents the performance of a simulation run in \Cref{main:sb:performance}. 
 We ran another simulation with healer count = 8 to show that as healer count increases,
 performance differences increase.
 The error was described using a bar chart (\Cref{main:fig:stack-plot-error}) and we 
 added a discussion regarding the raised issue in \Cref{main:sb:results}.
%{\color{red} Hence, we have deduced that the healer count curves
% give only a qualitative measure of system performance. 
% So a few deviations in plots does not mean that one configuration is better than another. 
% Indeed, if we check the results using the error metric in the experiment with healer count = 6,
% p = 1 perform substantially equal to p = 0.75. 
% But, increasing healer count (therefore having an increasing need of collective choices) p = 1 
% starts to deviate considerably from p = 0.75 configuration (both in qualitative and quantitative trend).
%}
}
%\say{MV}{ALL}{I would add a very brief description of what happens actually (3-4 lines)}
%\say{RC}{MV}{Do you meant in the paper or in the response? I think the former. @GA the red part is more a discussion than a description of the ``action'' to address the comment (which is better to keep short).}

%\bibliographystyle{elsarticle-num}
%\bibliography{bibliography}

\end{document}
