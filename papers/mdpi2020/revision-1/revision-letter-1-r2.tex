\documentclass{article} 
\usepackage[usenames,dvipsnames,svgnames,table]{xcolor}
\usepackage{hyperref}
\usepackage[usenames]{xcolor}
\usepackage{graphicx}
\usepackage{amsmath}
\usepackage{listings}
\usepackage{nameref}

\RequirePackage[sort&compress,sectionbib,numbers]{natbib}


\def\BibTeX{{\rm B\kern-.05em{\sc i\kern-.025em b}\kern-.08em
    T\kern-.1667em\lower.7ex\hbox{E}\kern-.125emX}}

\usepackage{xr}
\externaldocument[main:]{../paper20-mdpi-jsan-si-autonomy}
\usepackage{xr-hyper}


\newcounter{reviewer}
\newcounter{comment}[reviewer]

\setcounter{reviewer}{-1} % start with zero
\newcommand{\reviewer}{
	\subsection*{\refstepcounter{reviewer}Reviewer \arabic{reviewer}}
}
\newcommand{\reviewerl}[1]{
	\subsection*{\refstepcounter{reviewer}\label{#1}Reviewer \arabic{reviewer}}
}
\newcommand{\reviewerlt}[2]{
	\subsection*{\refstepcounter{reviewer}\label{#1}#2}
}
\newcommand{\comment}[1]{
	\subsubsection*{\refstepcounter{comment}Reviewer comment \arabic{comment}} %\arabic{reviewer}.
	\colorbox{gray!10}{\parbox[t]{\linewidth}{\setlength{\parskip}{0.5\baselineskip}%
 #1 }}
}
\newcommand{\rcref}[2]{\ref{#1}.\ref{#2}}
\newcommand{\commref}[1]{\ref{#1}}
\newcommand{\commentl}[2]{
	\subsubsection*{\refstepcounter{comment}\label{#1}Comment \arabic{comment}} %\arabic{reviewer}.
	\colorbox{gray!10}{\parbox[t]{\linewidth}{ #2 }}
}
\newcommand{\edcomment}[1]{
	\subsubsection*{Comment}
	\colorbox{gray!10}{\parbox[t]{\linewidth}{ #1 }}
}
\newcommand{\reply}[1]{	\\[2pt]
	\textbf{Author response:} 	
	#1
}
\newcommand{\action}[1]{	\\[2pt]
	\textbf{Action to address the comment:} 
	#1
}
\newcommand{\corrstart}{\color{red}}
\newcommand{\corrend}{\color{black}}
\newcommand{\correction}[1]{\corrstart #1\corrend{}}
\newcommand{\checkstart}{\color{violet}}
\newcommand{\tocheck}[1]{\checkstart{}#1\corrend{}}

\makeatletter
\renewcommand\p@comment{\thereviewer.}
\makeatother

\newcommand{\say}[3]{
\noindent\fcolorbox{black}{cyan!20!white}{
\begin{minipage}{\textwidth}
\textbf{{#1}} \textbf{({@}#2)}: #3
\end{minipage}
}
}


\usepackage{cleveref}


\begin{document}

\title{{\Large Revision Letter for} ``A Programming Approach to Collective Autonomy''}
\author{
Roberto Casadei
\and
Gianluca Aguzzi
\and
Mirko Viroli
}

\maketitle

Dear Reviewer, \newline

Thank you very much for your time and valuable feedback.
%
We have taken into account all your questions and suggestions and implemented corresponding revisions to our manuscript.
%
Our replies and corrective actions for your comments are detailed below.
%
Based on your comments and the comments raised by other Reviewers, we have implemented the following main revisions:
%
\begin{itemize}
\item we have added a comparison with respect to related work (\Cref{main:s:summary-comparison-rw});
\item we have clarified the contribution
(\Cref{main:s:intro}, \Cref{main:s:conc});
\item we have run more experiments and clarified aspects of the evaluation (\Cref{main:s:eval});
\item we have fixed several minor issues to further improve the overall quality of the manuscript.
\end{itemize}
%
For the Reviewer's convenience, we coloured in {\color{blue}blue} the parts of the paper that have been added or modified.
%
We hope that you will be satisfied with the revised version of our paper and consider it now suitable for publication.
\\ ~ \\

\noindent Sincerely yours,

Gianluca Aguzzi, Roberto Casadei, Mirko Viroli


\raggedbottom

%\section*{Comment and replies}

%\reviewerlt{r0}{Editor's Comments}
%\edcomment{
%}
%\reply{
%}

% (x) English language and style are fine/minor spell check required

%	Yes	Can be improved	Must be improved	Not applicable
%Does the introduction provide sufficient background and include all relevant references?
%(x)	( )	( )	( )
%Is the research design appropriate?
%(x)	( )	( )	( )
%Are the methods adequately described?
%( )	(x)	( )	( )
%Are the results clearly presented?
%( )	(x)	( )	( )
%Are the conclusions supported by the results?
%( )	( )	( )	( )

\reviewerlt{r2}{Comments and Replies}

\comment{
The proposed manuscript is interesting and the current situation in the field of research is descripted thoroughly, but several drawbacks are to be corrected.
}
\reply{
%
We thank the Reviewer for the provided comments,
 which have allowed us to further improve the manuscript.
}

\comment{
The reasons to apply namely ScaFi language for aggregate programming are not disclosed, as well as its possible advantages or shortcomings in comparison with other known languages.
}
\reply{
The reasons for using ScaFi w.r.t. other aggregate programming languages are mostly practical (since all these languages share the same formal underpinnings: the field calculus---or a variant thereof): the authors are more expert in it,
 and arguably it provides various benefits for both programming 
 (e.g., in terms of typing, reuse of Scala features and tooling)
  and presentation (e.g., through typing annotations, consistency in syntax/semantics, potential familiarity).
%
}
\action{
We have extended \Cref{main:background-ac} with a paragraph explaining why we adopted ScaFi instead of other aggregate programming languages.
}

\comment{
2. Numerical experiments, given for coordinated agents in wildlife modelling tasks are adequate, but are not discussed enough for specialists from adjacent and other research fields. Briefly general model task is to be described, also some comparison with similar papers is to disclosed, concerning the impact of different roles of agents and danger sensors.
}
\reply{
%We're in accord with the reviewer. 
% Indeed, in the case study, we haven't discussed enough concerns about tasks,
% roles and danger sense.  
%
%Our main focus is on collective behaviour
% and not on domain-specific aspects.
%
Thanks for the suggestion. %pointing out that tasks, roles, and sensors could be better described.
}
\action{
We have expanded \Cref{main:sb:experiment-setup} with a brief task model description. 
 We have added \Cref{main:sb:case-study-remarks} where we discuss 
 the impact of roles and danger sensing and some related work 
 in wildlife monitoring.
}

\comment{
3. The term “potential field” has very specific meaning in physics for real sensors and actuators, but appropriate commentaries concerning “computational fields” are scattered in the text. It is preferable to concentrate them in one of initial sections.
}
\reply{
We agree with the Reviewer that we did not explain this clearly.
%
The point is that a ``computational field'' (namely a distributed data structure that associates a value to each node of the network),
 such as for instance a ``gradient field'' (namely the computational field that associates any device with a corresponding distance to a source device),
 can be used to represent a ``potential field''
 (namely, a data structure that represents a concept similar to the corresponding notion in physics).
%
The only difference is that whereas potential fields in physics and robotics are generally a function of space,
 potential fields in aggregate computing 
 are a function of devices (which can in principle approximate space).
%
}
\action{
We have clarified the notion of a ``potential field'' 
 (as a particular kind of computational field)
 in the context of the example in \Cref{main:contrib-collective-autonomy} where it is first mentioned.
}

%\bibliographystyle{elsarticle-num}
%\bibliography{bibliography}

\end{document}
