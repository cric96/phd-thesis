\documentclass{article} 
\usepackage[usenames,dvipsnames,svgnames,table]{xcolor}
\usepackage{hyperref}
\usepackage[usenames]{xcolor}
\usepackage{graphicx}
\usepackage{amsmath}
\usepackage{listings}
\usepackage{nameref}

\RequirePackage[sort&compress,sectionbib,numbers]{natbib}


\def\BibTeX{{\rm B\kern-.05em{\sc i\kern-.025em b}\kern-.08em
    T\kern-.1667em\lower.7ex\hbox{E}\kern-.125emX}}

\usepackage{xr}
\externaldocument[main:]{../paper20-mdpi-jsan-si-autonomy}
\usepackage{xr-hyper}


\newcounter{reviewer}
\newcounter{comment}[reviewer]

\setcounter{reviewer}{-1} % start with zero
\newcommand{\reviewer}{
	\subsection*{\refstepcounter{reviewer}Reviewer \arabic{reviewer}}
}
\newcommand{\reviewerl}[1]{
	\subsection*{\refstepcounter{reviewer}\label{#1}Reviewer \arabic{reviewer}}
}
\newcommand{\reviewerlt}[2]{
	\subsection*{\refstepcounter{reviewer}\label{#1}#2}
}
\newcommand{\comment}[1]{
	\subsubsection*{\refstepcounter{comment}Reviewer comment \arabic{comment}} %\arabic{reviewer}.
	\colorbox{gray!10}{\parbox[t]{\linewidth}{\setlength{\parskip}{0.5\baselineskip}%
 #1 }}
}
\newcommand{\rcref}[2]{\ref{#1}.\ref{#2}}
\newcommand{\commref}[1]{\ref{#1}}
\newcommand{\commentl}[2]{
	\subsubsection*{\refstepcounter{comment}\label{#1}Comment \arabic{comment}} %\arabic{reviewer}.
	\colorbox{gray!10}{\parbox[t]{\linewidth}{ #2 }}
}
\newcommand{\edcomment}[1]{
	\subsubsection*{Comment}
	\colorbox{gray!10}{\parbox[t]{\linewidth}{ #1 }}
}
\newcommand{\reply}[1]{	\\[2pt]
	\textbf{Author response:} 	
	#1
}
\newcommand{\action}[1]{	\\[2pt]
	\textbf{Action to address the comment:} 
	#1
}
\newcommand{\corrstart}{\color{red}}
\newcommand{\corrend}{\color{black}}
\newcommand{\correction}[1]{\corrstart #1\corrend{}}
\newcommand{\checkstart}{\color{violet}}
\newcommand{\tocheck}[1]{\checkstart{}#1\corrend{}}

\makeatletter
\renewcommand\p@comment{\thereviewer.}
\makeatother

\newcommand{\say}[3]{
\noindent\fcolorbox{black}{cyan!20!white}{
\begin{minipage}{\textwidth}
\textbf{{#1}} \textbf{({@}#2)}: #3
\end{minipage}
}
}


\usepackage{cleveref}


\begin{document}

\title{{\Large Revision Letter for} ``A Programming Approach to Collective Autonomy''}
\author{
Roberto Casadei
\and
Gianluca Aguzzi
\and
Mirko Viroli
}

\maketitle

Dear Reviewer, \newline

Thank you very much for your time and valuable feedback.
%
We have taken into account all your questions and suggestions and implemented corresponding revisions to our manuscript.
%
Our replies and corrective actions for your comments are detailed below.
%
Based on your comments and the comments raised by other Reviewers, we have implemented the following main revisions:
%
\begin{itemize}
\item we have added a comparison with respect to related work (\Cref{main:s:summary-comparison-rw});
\item we have clarified the contribution
(\Cref{main:s:intro}, \Cref{main:s:conc});
\item we have run more experiments and clarified aspects of the evaluation (\Cref{main:s:eval});
\item we have fixed several minor issues to further improve the overall quality of the manuscript.
\end{itemize}
%
For the Reviewer's convenience, we coloured in {\color{blue}blue} the parts of the paper that have been added or modified.
%
We hope that you will be satisfied with the revised version of our paper and consider it now suitable for publication.
\\ ~ \\

\noindent Sincerely yours,

Gianluca Aguzzi, Roberto Casadei, Mirko Viroli

\raggedbottom

%\section*{Comment and replies}

%\reviewerlt{r0}{Editor's Comments}
%\edcomment{
%}
%\reply{
%}

% English: (x) I don't feel qualified to judge about the English language and style

%Yes	Can be improved	Must be improved	Not applicable
%--------------------------------------------------------
%Does the introduction provide sufficient background and include all relevant references?
%(x)	( )	( )	( )
%Is the research design appropriate?
%(x)	( )	( )	( )
%Are the methods adequately described?
%(x)	( )	( )	( )
%Are the results clearly presented?
%(x)	( )	( )	( )
%Are the conclusions supported by the results?
%(x)	( )	( )	( )


\reviewerlt{r1}{Comments and Replies}

\comment{
This paper presents an approach to programming individual and collective autonomous systems. The main idea is to combine the aggregate programming paradigm with agent-based architectures. The approach is exemplified with a simulated case study of wildlife monitoring. The paper also identifies gaps in the area of programming collective autonomy and points at future research avenues.

One of the key arguments of the authors is that aggregate programming is a suitable approach to programming multi-agent systems. This makes perfect sense as aggregate programming has been successful in many domains featuring e.g. autonomy of computing entities (agents) or the necessity of conciliating micro/agent views with macro/global views.


The paper is well written and organized. It provides a nice, short overview of the areas of interest (agents, autonomic computing, aggregate programming) and it describes the proposed approach. The experimental section serves to see the approach at work (on a simulated system).


I recommend accepting the paper
}
\reply{
The Reviewer's summary is perfect.
We are pleased that the Reviewer appreciated our work.
}

\comment{
Minor comments

- In Section 1, having the contributions and then the structure of the paper is redundant. I suggest merging both parts.
}
\action{
Since the contributions match the structure of the manuscript, we have merged both parts as suggested.
}


%\bibliographystyle{elsarticle-num}
%\bibliography{bibliography}

\end{document}
