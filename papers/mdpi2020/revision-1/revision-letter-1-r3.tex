\documentclass{article} 
\usepackage[usenames,dvipsnames,svgnames,table]{xcolor}
\usepackage{hyperref}
\usepackage[usenames]{xcolor}
\usepackage{graphicx}
\usepackage{amsmath}
\usepackage{listings}
\usepackage{nameref}

\RequirePackage[sort&compress,sectionbib,numbers]{natbib}


\def\BibTeX{{\rm B\kern-.05em{\sc i\kern-.025em b}\kern-.08em
    T\kern-.1667em\lower.7ex\hbox{E}\kern-.125emX}}

\usepackage{xr}
\externaldocument[main:]{../paper20-mdpi-jsan-si-autonomy}
\usepackage{xr-hyper}


\newcounter{reviewer}
\newcounter{comment}[reviewer]

\setcounter{reviewer}{-1} % start with zero
\newcommand{\reviewer}{
	\subsection*{\refstepcounter{reviewer}Reviewer \arabic{reviewer}}
}
\newcommand{\reviewerl}[1]{
	\subsection*{\refstepcounter{reviewer}\label{#1}Reviewer \arabic{reviewer}}
}
\newcommand{\reviewerlt}[2]{
	\subsection*{\refstepcounter{reviewer}\label{#1}#2}
}
\newcommand{\comment}[1]{
	\subsubsection*{\refstepcounter{comment}Reviewer comment \arabic{comment}} %\arabic{reviewer}.
	\colorbox{gray!10}{\parbox[t]{\linewidth}{\setlength{\parskip}{0.5\baselineskip}%
 #1 }}
}
\newcommand{\rcref}[2]{\ref{#1}.\ref{#2}}
\newcommand{\commref}[1]{\ref{#1}}
\newcommand{\commentl}[2]{
	\subsubsection*{\refstepcounter{comment}\label{#1}Comment \arabic{comment}} %\arabic{reviewer}.
	\colorbox{gray!10}{\parbox[t]{\linewidth}{ #2 }}
}
\newcommand{\edcomment}[1]{
	\subsubsection*{Comment}
	\colorbox{gray!10}{\parbox[t]{\linewidth}{ #1 }}
}
\newcommand{\reply}[1]{	\\[2pt]
	\textbf{Author response:} 	
	#1
}
\newcommand{\action}[1]{	\\[2pt]
	\textbf{Action to address the comment:} 
	#1
}
\newcommand{\corrstart}{\color{red}}
\newcommand{\corrend}{\color{black}}
\newcommand{\correction}[1]{\corrstart #1\corrend{}}
\newcommand{\checkstart}{\color{violet}}
\newcommand{\tocheck}[1]{\checkstart{}#1\corrend{}}

\makeatletter
\renewcommand\p@comment{} %{\thereviewer.}
\makeatother

\newcommand{\say}[3]{
\noindent\fcolorbox{black}{cyan!20!white}{
\begin{minipage}{\textwidth}
\textbf{{#1}} \textbf{({@}#2)}: #3
\end{minipage}
}
}

\usepackage{cleveref}



\begin{document}

\title{{\Large Revision Letter for} ``A Programming Approach to Collective Autonomy''}
\author{
Roberto Casadei
\and
Gianluca Aguzzi
\and
Mirko Viroli
}

\maketitle

Dear Reviewer, \newline

Thank you very much for your time and valuable feedback.
%
We have taken into account all your questions and suggestions and implemented corresponding revisions to our manuscript.
%
Our replies and corrective actions for your comments are detailed below.
%
Based on your comments and the comments raised by other Reviewers, we have implemented the following main revisions:
%
\begin{itemize}
\item we have added a comparison with respect to related work (\Cref{main:s:summary-comparison-rw});
\item we have clarified the contribution
(\Cref{main:s:intro}, \Cref{main:s:conc});
\item we have run more experiments and clarified aspects of the evaluation (\Cref{main:s:eval});
\item we have fixed several minor issues to further improve the overall quality of the manuscript.
\end{itemize}
%
For the Reviewer's convenience, we coloured in {\color{blue}blue} the parts of the paper that have been added or modified.
%
We hope that you will be satisfied with the revised version of our paper and consider it now suitable for publication.
\\ ~ \\

\noindent Sincerely yours,

Gianluca Aguzzi, Roberto Casadei, Mirko Viroli

\raggedbottom

%\section*{Comment and replies}

%\reviewerlt{r0}{Editor's Comments}
%\edcomment{
%}
%\reply{
%}

% (x) I don't feel qualified to judge about the English language and style

%	Yes	Can be improved	Must be improved	Not applicable
%Does the introduction provide sufficient background and include all relevant references?
%( )	(x)	( )	( )
%Is the research design appropriate?
%( )	( )	(x)	( )
%Are the methods adequately described?
%( )	( )	(x)	( )
%Are the results clearly presented?
%( )	( )	(x)	( )
%Are the conclusions supported by the results?
%( )	(x)	( )	( )

\reviewerlt{r3}{Comments and Replies}

\comment{
In my view, the article has two distinct parts. An adequate introduction in which the problem under study is focused and adequately explained, and a development that needs to be improved in depth. I would suggest that the authors consider these recommendations:
}
\reply{
We have carefully considered the Reviewer's recommendations as detailed below.
}

\comment{
- The resolution of some figures (e.g., Figure 3 and Figure 4) should be improved.
}
\action{
\begin{itemize}
\item We have updated the original Figure 3 (now it is \Cref{main:fig:wildlife-monitoring-gui}) by generating another simulation snapshot and enlarging the graphical elements therein to make the figure sharper overall.
\item We have updated the original Figure 4 (now it is \Cref{main:fig:wildlife-simulation}) using PDF images (vector graphics), hence improving its quality.
\end{itemize} 
}

\comment{
- 4. Case study I miss the description of why the given experiment from 2013 was chosen for the given purposes. What are the advantages for us compared to other possible potential simulations. What was crucial for us there ?
}
\reply{
Though the Reviewer refers to an ``experiment from 2013'' and ``simulations'', we believe that the Reviewer actually wanted an explanation about the adopted \emph{simulator}, namely Alchemist---indeed, Alchemist is the subject of the only article of 2013 that had been cited.

On one hand, the emphasis of our evaluation is not on the simulator \emph{per se},
 but rather on the behaviour of an aggregate system described through ScaFi.
%
On the other hand, 
 the simulator of choice should
 provide a support for simulating
 logical networks of devices
 with a dynamics 
 suitable to implement
 the aggregate execution protocol
 discussed in \Cref{main:contrib-ac-control-arch}.
%
We chose Alchemist because
 it already comes with a support 
 for aggregate programming,
 i.e., 
 it comprises a module that enables 
 running (ScaFi) aggregate programs
 according to the discussed aggregate execution protocol.
%
Furthermore, it is a well-consolidated software used in literature 
 to realise several simulations about multi-agent and pervasive systems. 
%
%For simulations, we could have used whatever simulator as a matter of principle. In fact, our focus in this article is mainly on the paradigm (aggregate computing)  and not on the simulation system. We choose this platform primarily because it supports 
% the aggregate computing paradigm. Furthermore, it is a well-consolidated software used in literature in various simulations about pervasive systems. 
%
}
\action{At the very beginning of \Cref{main:s:eval}, we have added more details about simulator requirements as well as about the chosen simulator, Alchemist.}

\commentl{r3-rw}{
- Overall, I lack a comparison with the alternative approaches currently used in the article. What do the authors see as their contribution, contribution in the given area.
}
\reply{We agree with the reviewer that we did not explicitly compare the proposed approach with other related works (such as those reviewed in Section 2), and that such a discussion would allow us to better delineate our contribution.}
\action{
We have added a new subsection, \Cref{main:s:summary-comparison-rw}, 
 discussing how the proposed approach relates to other representative works 
 (i.e., AgentSpeak(R)/Jason, MOISE, and SCEL).
}

\comment{
- I miss the description of real scenarios of using MASs in real practice. Its importance for the future is not described in more detail.
}
\reply{We warmly thank the Reviewer for pointing out this missing part. This also allows us to further discuss the significance and positioning of the work within the research and technological landscape.
}
\action{
We have added a new subsection, \Cref{main:roadmap:apps},
 covering traditional as well as forthcoming MAS applications.
}

\commentl{r3-on-goals}{
- In the introduction, the authors set goals:%
\emph{\begin{itemize}
\item ``we provide a review of literature about autonomy and especially collective autonomy in MASs;
\item we analyse the aggregate computing framework by the perspective of autonomy, by covering its positioning w.r.t individual and collective autonomy, and showing how it can supports adjustable autonomy;
\item we exemplify the discussion through a simulated case study, investigating (i) the relationship between individual goals/autonomy and collective goals/autonomy; and (ii) the relationship between structures and collective autonomy; and
\item we provide gaps in literature on programming reliable collective autonomy and delineate a research roadmap.''
\end{itemize}}
%
At the end of the article, however, I lack a more specific description of how they met these goals and whether they are in line with their expectations.
}
\reply{The Reviewer is right that a more explicit retrospective is needed. Moreover, we note that this is also instrumental for better specifying the contributions and what is instead meant as a future work.}
\action{
We have largely extended the Conclusion with an additional paragraph that discusses explicitly the contributions w.r.t. the goals pointed out in the Introduction.
}

\comment{
- What do the authors see as their scientific contribution to this article?
}
\reply{
On one hand, 
 we contribute on aggregate computing research
 by discussing 
 how the framework relates to and supports various forms of autonomy.
%
This also includes viewing the aggregate execution protocol
 as an agent control architecture (cf. \Cref{main:fig:aggregate-control-arch} in the manuscript).
%
On the other hand, 
 we contribute on multi-agents systems research
 by showing 
 how -- through the aggregate computing paradigm --
 a collection of agents 
 can be programmed 
 towards collective tasks 
 using global-level scripts
 where the degree of individual vs. collective autonomy 
 can be adjusted.
%
In particular, the approach is alternative 
 to other state-of-the-art approaches
 as discussed in \Cref{main:s:summary-comparison-rw}.
%
Another contribution 
 lies in raising the problem of \emph{programming}
 collective autonomous behaviour,
 and proposing directions for further research:
 though there exist studies on collective autonomy,
 to the best of our knowledge
 no practical approaches have been proposed yet,
 and we expect the issue to become more significant in the future,
 with the advent of cyber-physical collectives.
}
\action{
We have clarified the contribution through the following interventions:
\begin{itemize}
\item addition of a new subsection, \Cref{main:s:summary-comparison-rw}, comparing the proposed approach with related work (cf. action for Comment~\ref{r3-rw});
\item addition of a new paragraph in the Conclusion (\Cref{main:s:conc}) commenting on the goals stated in the Introduction (cf. action for Comment~\ref{r3-on-goals}); and
\item addition of an explicit statement at the end of the Introduction.
\end{itemize}
}




%\bibliographystyle{elsarticle-num}
%\bibliography{bibliography}

\end{document}
