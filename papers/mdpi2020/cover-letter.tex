\documentclass{article} 
\usepackage[usenames,dvipsnames,svgnames,table]{xcolor}
\usepackage[hidelinks]{hyperref}
\usepackage[usenames]{xcolor}

\newcommand{\meta}[1]{\textcolor{blue}{#1}}

\newcommand{\theTitle}{A Programming Approach to Collective Autonomy}
\newcommand{\thePublisher}{MDPI}
\newcommand{\theJournal}{Sensor and Actuator Networks}
\newcommand{\theSI}{Agents and Robots for Reliable Engineered Autonomy}

\title{Cover Letter for the Article\\
``\theTitle''}
\author{Roberto Casadei \and Gianluca Aguzzi \and Mirko Viroli}

\usepackage[backend=bibtex]{biblatex}
\bibliography{bibliography}

\sloppypar

\begin{document}
\maketitle

Dear \emph{Editors},

we would like to submit our manuscript entitled ``\theTitle{}''
to the \thePublisher{} Journal on \theJournal{}, Special Issue on ``\theSI{}''.



%We believe this work makes for a great contribution to your special issue. 
%%
%Specifically, 
%it closely matches the following Topics of Interest reported in the Call for Papers:
%%
%\begin{itemize}
%\item[T1)] Nature-inspired hybrid SC methods for intelligence edge paradigm
%\item[T2)] Swarm Intelligence based algorithms for edge system
%\item[T3)] Software and simulation platform for SC in edge paradigm
%\item[T4)] SC for autonomic resource management in edge computing
%\end{itemize}
%%
%Indeed, this work contributes to the methodology and practice
% of \emph{aggregate computing}~\cite{DBLP:journals/computer/BealPV15,DBLP:journals/jlap/ViroliBDACP19}, an approach to collective adaptive systems development that  builds on the physics-inspired notion of computational fields~\cite{DBLP:journals/tocl/AudritoVDPB19} (cf. T1)
% and accordingly enables functional, compositional specification of swarm/collective intelligence~\cite{DBLP:journals/eaai/CasadeiVAPD21} (cf. T2).
%%
%In particular, we consider the problem
% of aggregate application deployment~\cite{DBLP:journals/isci/CasadeiFPRSV19,DBLP:journals/fgcs/CasadeiFPRSV19}
% as well as the recently introduced notion of \emph{pulverisation} of aggregate systems~\cite{DBLP:journals/fi/CasadeiPPVW20}
% and develop a \emph{methodology} and simulation-based \emph{toolchain} (cf. T4)
% for estimating in advance the deployment performance
% of intelligent collective edge services
% developed using this paradigm.
%%
%We evaluate the approach through a case study
% (originally introduced in~\cite{DBLP:journals/fgcs/PianiniCVN21}) 
% where an edge ecosystem self-organises into clusters to promote load balancing (cf. T4).
%%
%Given the relevance to the special issue, 
% the significance of the contribution,
% and the quality of the presentation,
% we believe this manuscript
% well fits this venue.

In this manuscript,
 we address the theme of \emph{autonomy} in multi-agent systems,
 and especially focus on \emph{collective} autonomy,
 namely the form of autonomy that emerges at the level of a collective (group of individuals).
%
In particular, 
 we consider the aggregate computing framework~\cite{DBLP:journals/computer/BealPV15,viroli2019jlamp-si-coord},
 discussing and showcasing through simulation 
 how it supports
 the \emph{programming} of self-organising multi-agent systems
 exhibiting various forms of autonomy.
%
%Moreover, we also review related work and discuss open research directions.
%
By virtue of the addressed themes, 
 we believe the contribution fits this special issue very well.

We also hereby declare that:
%
\begin{itemize}
\item This manuscript is the authors' original work and has not been published nor has it been submitted simultaneously elsewhere.
\item All authors have checked the manuscript and have agreed to the submission. 
\end{itemize}

Sincerely yours,\\
Gianluca Aguzzi, Roberto Casadei, Mirko Viroli.


\printbibliography

\end{document}
