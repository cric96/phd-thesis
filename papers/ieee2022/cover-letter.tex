\documentclass{article} 
\usepackage[usenames,dvipsnames,svgnames,table]{xcolor}
\usepackage[hidelinks]{hyperref}
\usepackage[usenames]{xcolor}

\newcommand{\meta}[1]{\textcolor{blue}{#1}}

\newcommand{\theTitle}{Dynamic Decentralization Domains for the Internet of Things}
\newcommand{\thePublisher}{IEEE}
\newcommand{\theJournal}{Internet Computing}
\newcommand{\theSI}{Decentralized Systems}

\title{Cover Letter for the Article\\
\textbf{``\theTitle''}
\\
submitted to
\\
\textbf{\thePublisher{} \theJournal{}\\ Special Issue on \theSI{}}}
\author{Gianluca Aguzzi \and Roberto Casadei \and Danilo Pianini \and Mirko Viroli}

\bibliographystyle{elsarticle-num}

\sloppypar

\begin{document}
\maketitle

Dear \emph{Editors},

we would like to submit our manuscript entitled ``\theTitle{}''
to the \thePublisher{} \theJournal{} Magazine.

We believe this work makes for a significant contribution to the journal. 
%
Indeed,
 in this manuscript 
 we propose a high-level programming model
 for decentralized situation recognition and action.
%
The idea consists 
 in exploiting a small number of 
 design abstractions,
 inspired by the state of the art~\cite{DBLP:journals/computer/BealPV15,DBLP:journals/fgcs/PianiniCVN21,DBLP:journals/eaai/CasadeiVAPD21,DBLP:journals/tetc/ScekicSVRTMD20,DBLP:journals/computer/BuresPKTH16,DBLP:journals/csur/MottolaP11}
 and integrated in order to express the global-level logic
 of self-organizing ecosystems, such that it can be executed in a fully decentralized way.
%
Specifically, we propose to combine concurrent collective tasks
 and decentralization domains
 to structure 
 the behaviour
 of systems
 e.g. for 
 monitoring and controlling
 large-scale environmental phenomena.
%
We implement the programming model 
 through a Scala API
 and evaluate it by simulation
 through a case study 
 of a large-scale flood warning system.
%
The results show the validity of the contribution for 
 designers and programmers
 of complex adaptive systems
 like those found e.g. in the Internet of Things and related scenarios.

Given the relevance to the journal's audience, 
 the significance of the contribution,
 and the vision we provide on emerging IoT scenarios,
 we believe this manuscript
 well fits this venue.

We also hereby declare that:
%
\begin{itemize}
\item This manuscript is the authors' original work and has not been published nor has it been submitted simultaneously elsewhere.
\item All authors have checked the manuscript and have agreed to the submission. 
\end{itemize}

Sincerely yours,\\
Gianluca Aguzzi, Roberto Casadei, Danilo Pianini, and Mirko Viroli.


\bibliography{bibliography}

\end{document}
